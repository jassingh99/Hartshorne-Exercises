\label{1.4.4}

A variety $Y$ is \emph{rational} if it is birationally equivalent to $\P^n$ for some $n$ (or, equivalently by \cite[I.4.5]{hartshorne}, if $k(Y)$ is a purely transcendental extension of $k$).

\begin{enumerate}[label = (\alph*)]
    \item Any conic in $\P^2$ is a rational curve.

    \item THe cuspidal cubic $y^2 = x^2$ is a rational curve.

    \item Let $Y$ be the nodal cubic curve $y^2 z = x^2 (x + z)$ in $\P^2$. Show that the projection $\phi$ from the point $P = [0 : 0 : 1]$ to the line $z = 0$ (Exc. \ref{1.3.14}) induces a birational map from $Y$ to $\P^1$. Thus, $Y$ is a rational curve.
\end{enumerate}

\begin{proof}

\end{proof}
