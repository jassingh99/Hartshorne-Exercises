\label{1.4.9}

Let $X$ be a projective variety of dimension $r$ in $\P^n$ with $n \geq r + 2$. Show that for a suitable choice of $P \notin X$, and a linear $\P^{n - 1} \subseteq \P^n$, the projection from $P$ to $\P^{n - 1}$ (Exc. \ref{1.3.4}) induces a \emph{birational} morphism of $X$ onto its image $X' \subseteq \P^{n - 1}$. You will need to use \cite[I.4.6A, I.4.7A, I.4.8A]{hartshorne}. This shows in particular that the birational map of \cite[I.4.9]{hartshorne} can be obtained by a finite number of such projections.

\begin{proof}
    Well let's think about how this would affect function fields. Though I still don't know why $X'$ is even a variety. Let's imagine $[0 : \dots : 0 : 1] \notin X$ and let $\P^{n-1} = Z(x_n)$. Then the projection $\phi: \P^n - P \longrightarrow \P^{n - 1}$ is given by $[a_0 : \dots : a_n] \mapsto [a_0 : \dots : a_{n - 1}]$. This is finite to one and onto (see \ref{1.4.8}) so it is of course dominant. Thus it induces a map on function fields $k(\P^{n - 1}) \longrightarrow k(\P^n)$ via $(U, f) \mapsto (\phi^{-1}[U], f \circ \phi)$. We can write elements of $k(\P^{n - 1})$ as rational functions $\frac{f(x_0, \dots, x_{n-1})}{g(x_0), \dots, x_{n-1}}$ which are homogeneous of the same degree. Then similarly, $\frac{f}{g} \circ \phi$ can be written as a homogeneous element of $k(x_0, \dots, x_n)$ of degree $0$. . This then sends $[a_0 : \dots : a_n]$ to $\frac{f(a_0, \dots, a_{n-1})}{g(a_0, \dots, a_{n-1})}$, so the induced map $k(\P^{n - 1}) \longrightarrow k(\P^n)$ is just the inclusion $k(x_0, \dots, x_{n-1})^0 \longrightarrow k(x_0, \dots, x_n)^0$.

    How about on $X \longrightarrow X'$? It's described in the same way as $(U, f) \mapsto (\phi^{-1}[U], f \circ \phi)$. Also, we can write $k(X) = S(X)_{((0))}$ the homogeneous localization, and analogously for $X'$ (if it's even a variety??). For this to be birational, I'd hope $k(X') \longrightarrow k(X)$ is an isomorphism. But I am lost.
\end{proof}
