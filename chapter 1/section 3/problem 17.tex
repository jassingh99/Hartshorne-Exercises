\label{1.3.17}

\emph{Normal Varieties}

A variety $Y$ is \emph{normal at a point $P \in Y$} if $\mathcal O_P$ is an integrally closed ring. $Y$ is \emph{normal} if it is normal at every point.

\begin{enumerate}[label = (\alph*)]
    \item Show that every conic in $\P^2$ is normal.

    \item Show that the quadric suraces $Q_1, Q_2$ in $\P^3$ given by the equations $Q_1 = Z(xy - zw)$, $Q_2 = Z(xy - z^2)$ are normal. (cd. \ref{2.6.4} for the latter.)

    \item Show that the cuspidal cubic $y^2 = x^3$ in $\A^3$ is not normal.

    \item If $Y$ is affine, then $Y$ is normal iff $A(Y)$ is integrally closed.

    \item Let $Y$ be an affine variety. Show that there is a normal variety $\widetilde{Y}$, and a morphism $\pi: \widetilde{Y} \longrightarrow Y$, with the property that whenever $Z$ is a normal variety and $\phi: Z \longrightarrow Y$ is a \emph{dominant} morphism, (i.e. $\phi[Z]$ is dense in $Y$), then there is a unique morphism $\theta: Z \longrightarrow \widetilde{Y}$ such that $\phu = \pi \circ \theta$. $\widetilde{Y}$ is called the \emph{normalization} of $Y$. You will need \cite[I.3.9A]{hartshorne} above.
\end{enumerate}

\begin{proof}
    \begin{enumerate}[label = (\alph*)]
        \item By \ref{1.3.1}.c, all conics are isomorphic to $\P^1$. Furthermore, local rings are functorial (being colimits) so it suffices to show that $\P^1$ is normal. Take $P \in \P^1$. We can use the isomorphism $\mathcal O_P \cong S(Y)_{(\m_P)}$, but the degree 0 stipulation makes it a bit more annoying. An easier thing to do would be to take $P \in \A^1 \subseteq \P^1$ for any open embedded copy of $\A^1$. Then $\mathcal O_P$ can be computed in $\A^1$ or $\P^1$. It's easy to do this for $\A^1$, as then $\mathcal O_P \cong k[t]_{\m_P}$. $k[t]$ is a UFD therefore normal, and integral closure commutes with localization. Hence, the localization of a normal domain is normal, so $\mathcal O_P$ is normal.

        \item I haven't done these computations yet.

        \item This follows from part (d) of this problem, but I'll do it directly too. Above (\ref{fig1.3.1}) we have a picture of the cuspidal cubic. The messed up part (the cusp!) is at the origin, so we suspect that the local ring at the origin will not be integrally closed. Indeed, $\mathcal O_0 \cong A(X)_{\m_0}$, where $X$ is the cuspidal cubic $V(y^2 - x^3)$. The maximal ideal $\m_0$ is given by those polynomials which vanish at $0$, so it is generated by $(x, y)$. Note that we have the isomorphism $k[x, y]/(y^2 - x^3) \longrightarrow k[t^2, t^3]$ via $x \mapsto t^2$, $y \mapsto t^3$. Furthermore, this isomorphism sends $\m_0 = (x, y) \mapsto (t^2, t^3)$. Thus, $\mathcal O_0 \cong k[t^2, t^3]_{(t^2, t^3)}$. The quotient field of this ring is $k(t)$. Note that $t \in k(t)$ is integral over $k[t^2, t^3]_{(t^2, t^3)}$ but is not in the ring. Hence, $\mathcal O_0$ is not normal and thus, $X$ is not normal.

        \item $(\Longleftarrow)$ Integral closure commutes with localization and $\mathcal O_P \cong A(Y)_{\m_P}$.

        $(\Longrightarrow)$ Suppose that $\mathcal O_P$ is normal for all $P$. Then $A(Y)_{\m_P}$ is normal for all $P \in Y$. By correspondence and the Nullstellensatz, this means that for all $\m \in Max(A(Y))$, $A(Y)_\m$ is normal. It is a general fact of commutative algebra that $A(Y)$ is therefore normal. Indeed, let $A(Y) \subseteq R \subseteq k(Y)$ be the integral closure. Then by flatness of localization , we have $A(Y)_\m \subseteq R_\m \subseteq k(Y)$. As integral closure commutes with localization, the inclusion $R_\m$ is the integral closure of $A(Y)_\m$ in $k(Y)$. We assumed that $A(Y)_\m$ is integrally closed, so the inclusion $A(Y)_\m \subseteq R_\m$ is onto. Thus, the inclusion $A(Y) \subseteq R$ is onto at all localizations by maximal ideals. In other words, its cokernel is locally $0$, therefore $0$. Hence, it is onto and $A(Y) = R$ is normal.

        \item I haven't solved this yet. I suspect that the normalization will be defined as follows. Take $A(Y) \subseteq R \subseteq k(Y)$ the integral closure. Then by \cite[3.9A]{hartshorne}, $R$ is a finite type domain over $k$ and therefore corresponds to an affine variety $\widetilde{Y}$. Furthermore, the inclusion map $i: A(Y) \subseteq R$ yields a map $\pi: \widetilde Y \longrightarrow Y$.

        For example, take $Y = V(y^2 - x^3)$ the cuspidal cubic. Then its coordinate ring is $A(Y) = k[x, y]/(y^2 - x^3) \cong k[t^2, t^3]$, whose integral closure is $k[t]$. This corresponds to the map $\A^1 \longrightarrow Y$ via $t \mapsto (t^2, t^3)$.

        Anyways, it is a fact of commutative algebra that an integral extension like $i: A(Y) \subseteq A(\widetilde Y)$ yields a surjective map $\spec(A(\widetilde Y)) \longrightarrow \spec(A(Y))$, which reduces to a surjective map $Max(A(\widetilde Y)) \longrightarrow Max(A(Y))$. By the Nullstellensatz, this yields the map $\pi: \widetilde Y \longrightarrow Y$ which is therefore onto.

        I'm a bit lost here, so here are some facts:
        \begin{enumerate}[label=(\roman*)]
            \item A commutative diagram of varieties
            
            \[
                \begin{tikzcd}
                    \widetilde Y \arrow[r, "\pi"] & Y\\
                    Z \arrow[u, dashed, "\exists!"] \arrow[ur, "\phi", swap]
                \end{tikzcd}
            \]
            
            is the same as a commutative diagram of $k$-algebras

            \[
                \begin{tikzcd}
                    A(\widetilde Y) \arrow[d, dashed, "\exists!" swap] & A(Y) \arrow[l, "i", swap] \arrow[dl, "\Phi"]\\
                    \mathcal O(Z)
                \end{tikzcd}
            \]
            
            \item What do dominant maps on varieties look like on coordinate rings? One fact we know is that they induce injections on local rings $\mathcal O_{Y, \phi(P))} \longrightarrow \mathcal O_{Z, P}$ and hence yield an extension of function fields $k(Y) \longrightarrow k(Z)$.

            \item If we had a diagram of coordinate rings as in (i), then recall that every $a \in A(\widetilde Y)$ is the root of some monic polynomial $f \in A(Y)[t]$. Thus, $\Phi(f)(\Phi(a)) = 0$. So we need $\Phi(a)$ to be integral over $A(Y)$ with respect to the algebra structure induced by $\Phi$. That is, $\Phi$ must be an integral map. Is this the precise condition to get a dominant map of varieties? Does this give us the desired mapping property? The integral closure is the ``smallest" integral extension right? Is there a UP of integral closure?
        \end{enumerate}
    \end{enumerate}
\end{proof}                                                                                        
