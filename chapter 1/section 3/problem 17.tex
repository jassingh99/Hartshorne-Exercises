\label{1.3.17}

\emph{Normal Varieties}

A variety $Y$ is \emph{normal at a point $P \in Y$} if $\mathcal O_P$ is an integrally closed ring. $Y$ is \emph{normal} if it is normal at every point.

\begin{enumerate}[label = (\alph*)]
    \item Show that every conic in $\P^2$ is normal.

    \item Show that the quadric suraces $Q_1, Q_2$ in $\P^3$ given by the equations $Q_1 = Z(xy - zw)$, $Q_2 = Z(xy - z^2)$ are normal. (cd. \ref{2.6.4} for the latter.)

    \item Show that the cuspidal cubic $y^2 = x^2$ in $\A^3$ is not normal.

    \item If $Y$ is affine, then $Y$ is normal iff $A(Y)$ is integrally closed.

    \item Let $Y$ be an affine variety. Show that there is a normal variety $\widetilde{Y}$, and a morphism $\pi: \widetilde{Y} \longrightarrow Y$, with the property that whenever $Z$ is a normal variety and $\phi: Z \longrightarrow Y$ is a \emph{dominant} morphism, (i.e. $\phi[Z]$ is dense in $Y$), then there is a unique morphism $\theta: Z \longrightarrow \widetilde{Y}$ such that $\phu = \pi \circ \theta$. $\widetilde{Y}$ is called the \emph{normalization} of $Y$. You will need \cite[I.3.9A]{hartshorne} above.
\end{enumerate}

\begin{proof}

\end{proof}
