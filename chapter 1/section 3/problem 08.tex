\label{1.3.8}

Let $H_i$ and $H_j$ be the hyperplanes in $\P^n$ defined by $x_i = 0$ and $x_j = 0$ with $i \neq j$. Show that any regular function on $\P^n - (H_i \cap H_j)$ is constant. (This gives an alternate proof of \cite[I.3.4a]{hartshorne} in the case $Y = \P^n$.)

\begin{proof}
    We can write $\P^n - (H_i \cap H_j) = (\P^n - H_i) \cup (\P^n - H_j)$. Hence, $\mathcal O(\P^n - (H_i \cap H_j)) = \mathcal O(\P^n - H_i) \cap \mathcal O(\P^n - H_j)$. Recall that we write $U_i = \P^n - H_i$ and that we had an isomorphism of varieties $\phi_i: U_i \longrightarrow \A^n$. Indeed, since we know $\mathcal O(\A^n)$ is the polynomial ring over $k$ in $n$ variables, we can compute $U_i$. More precisely, take $\mathcal O(\A^n) = k[x_0, \dots, \widehat{x_i}, \dots, x_n]$. Then the isomorphism $\mathcal O(\A^n) \longrightarrow \mathcal O(U_i)$ takes $f(x_0, \dots, \widehat{x_i}, \dots, x_n) \mapsto \frac{\beta_i(f)}{x_i^{\deg f}}$, where $ \beta_i(f) = x_i^{\deg f} f\parens{\frac{x_0}{x_i}, \dots, \widehat{\frac{x_i}{x_i}}, \dots, \frac{x_n}{x_i}}$. Now, one maybe tempted to cancel the $x_i^{\deg f}$ from the numerator and denominator, but one must please control themself. The numerator $\beta_i(f)$ is a degree $\deg f$ homogeneous polynomial that is not divisible by $x_i$, as $f$ has no $x_i$ terms. Hence, keeping it as this quotient makes it readily obvious exactly how this is interpreted as a regular function on $\mathcal O(U_i)$.

    Now, take some $\alpha \in \mathcal O(U_i) \cap \mathcal O(U_j)$. Then we can write $\alpha = \frac{\beta_i(f)}{x_i^{\deg f}}$ and $\alpha = \frac{\beta_j(g)}{x_j^{\deg g}}$ for $f \in k[x_0, \dots, \widehat{x_i}, \dots, x_n]$ and $g \in k[x_0, \dots, \widehat{x_j}, \dots, x_n]$. Then, setting these two quotients equal, we get $x_j^{\deg g} \beta_i(f) = x_i^{\deg f} \beta_j(g)$. Now, as $i \neq j$, $x_i$ and $x_j$ are coprime. Hence, $x_i^{\deg f} | \beta_i(f)$. But as discussed above, no positive power of $x_i$ can divide $\beta_i(f)$. Hence, $\deg f = 0$ so $\frac{\beta_i(f)}{x_i^{\deg f}}$ is a constant, i.e. $\alpha$ is a constant. Thus, $\mathcal O(\P^n - (H_i \cap H_j)) = \mathcal O(U_i) \cap \mathcal O(U_j) = k$. Less formally, the idea is that if there was a positive power of $x_i$ in the denominator then this could not be defined on $U_j \cap H_i$, lest we divide by 0.
\end{proof}
