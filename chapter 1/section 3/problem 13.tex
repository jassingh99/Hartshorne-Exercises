\label{1.3.13}

\emph{The Local Ring of a Subvariety}

Let $Y \subseteq X$ be a subvariety. Let $\mathcal O_{Y, X}$ be the set of equivalence classes $(U, f)$ where $U \subseteq X$ is open, $U \cap Y \neq \emptyset$, and $f$ is a regular function of $U$. We say that $(U, f)$ is equivalent to $(V, g)$ if $f = g$ on $U \cap V$. Show that $\mathcal O_{Y, X}$ is a local ring, with residue field $K(Y)$ and dimension $= \dim X - \dim Y$. It is the \emph{local ring} of $Y$ on $X$. Note that if $Y = \curly{P}$ is a point we get $\mathcal O_P$, and if $Y = X$ we get $K(X)$. Note also that if $Y$ is not a point, then $K(Y)$ is not algebraically closed, so in this way we get local rings whose residue fields are not algebraically closed.

\begin{proof}
    We claim that the unique maximal ideal of $\mathcal O_{Y, X}$ is $\m_{Y} := \curly{(U, f) \in \mathcal O_X : f|_{Y} = 0}$, in analogy to $\m_P$. This is, of course, the kernel of the map $\mathcal O_{Y, X} \longrightarrow \mathcal O_Y \subseteq k(Y)$ by restriction.

    idea: easy to see that you can invert away from $Y$ so it's local. the lifting stuff from \ref{1.3.10} should give that the residue field is appropriate. idk about the dimension. we can embed the quotient field in $k(X)$, which has trdeg $\dim X$ over $k$ and the residue field is $k(Y)$ which has trdeg $\dim Y$ over $k$. can we subtract these?
\end{proof}
