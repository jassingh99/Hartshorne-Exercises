\label{1.3.1}

\begin{enumerate}[label = (\alph*)]
    \item Show that any conic in $\A^2$ is either isomorphic to $\A^1$ or $\A^1 - 0$ (c.f \ref{1.1.1}).

    \item Show that $\A^1$ is \emph{not} isomorphic to any proper open subset of itself. (This result is generalized by \ref{1.6.7} below).

    \item Any conic in $\P^2$ is isomorphic to $\P^1$.

    \item We will see later (\ref{1.4.8}) that any two curves are homeomorphic. But show now that $\A^2$ is not even homeomorphic to $\P^2$.

    \item If an affine variety is isomorphic to a projective variety, then it consists of only one point.
\end{enumerate}

\begin{proof}
    \begin{enumerate}[label = (\alph*)]
        \item A conic is the zero set of an irreducible polynomial $f$ of degree $2$. By the results of this section, it will be fruitful to consider the coordinate rings. Indeed, by \ref{1.1.1}.c, $A(V(f)) \cong k[x, y]/(y - x^2)$ or $k[x, y]/(xy - 1)$. The former is isomorphic to $k[t]$ via $x \mapsto t$, $y \mapsto t^2$, corresponding to the map $\A^1 \longrightarrow \A^2$ via $t \mapsto (t, t^2)$. The algebra isomorphism yields an isomorphism of varieties.

        In the latter case the obvious maps are $V(xy - 1) \longrightarrow \A^1 - 0$ via $(x, y) \mapsto x$ and $\A^1 - 0 \longrightarrow V(xy - 1)$ sending $t \mapsto (t, 1/t)$. The former is induced by the map $k[t] \longrightarrow k[x, y]/(xy - 1)$ sending $t \mapsto x$, and it of course has image equal to exactly $\A^1 - 0$. For the latter, we consider the map $k[x, y]/(xy - 1) \longrightarrow \mathcal O(\A^1 - 0)$ sending $x \mapto t$ and $y \mapsto 1/t$. As \cite[I.3.5]{hartshorne} was proven in this generality, this induces the map $\A^1 - 0 \longrightarrow V(xy - 1)$ we just described. It is easy to check that these are inverse to each other, and we have additionally shown that they are both maps of varieties via their rings of regular functions.

        \item Take $U < \A^1$. Of course, if $\emptyset = U$ then we are done so suppose otherwise. Then $Y = \A^1 - U$ satisfies $0 < Y < \A^1$. We can therefore write $Y = V(f)$ for some nonzero, nonconstant polynomial $f \in k[x]$. We can also assume WLOG that $f$ is squarefree. If indeed $\A^1 \cong U$, then $k[x] = \mathcal O(\A^1) \cong \mathcal O(U)$ as $k$-algebras. Consider then our hypothetical isomorphism $k[x] \cong \mathcal O(U)^\times$. As this is an isomorphism of $k$-algebras, we must have had the following commutative diagram.
        $$
        \begin{tikzcd}
            k[x] \arrow[rr, "\sim"] && \mathcal O(U)\\
            & k \arrow[ur] \arrow[ul]
        \end{tikzcd}
        $$

        This descends to the unit groups as follows.
        $$
        \begin{tikzcd}
            k^\times \arrow[rr, "\sim"] && \mathcal O(U)^\times\\
            & k^\times \arrow[ur] \arrow[ul]
        \end{tikzcd}
        $$

        Which would imply that $\mathcal O(U)^\times = k^\times$. But as $V(f) \cap U = \emptyset$, $f \in \mathcal O(U)^\times$. Furthermore, $f$ was assumed to be nonconstant, so this is a contradiction.
    \end{enumerate}
\end{proof}
