\label{1.3.16}

\emph{Products of Quasi-Projective Varieties}

Use the Segre embedding \ref{1.2.14} to identify $\P^n \times \P^m$ with its image and hence give it a structure of a projective variety. Now for any two quasi-projective varieties $X \subseteq \P^n$ and $Y \subseteq \P^m$, consider $X \times Y \subseteq \P^n \times \P^m$

\begin{enumerate}[label = (\alph*)]
    \item Show that $X \times Y$ is a quasi-projective variety.

    \item If $X, Y$ are both projective, show that $X \times Y$ is projective.

    \item Show that $X \times Y$ is a product in the category of varieties.
\end{enumerate}

\begin{proof}
    \begin{enumerate}[label = (\alph*)]
        \item In the course of proving (b), we will show that if $X, Y \subseteq \P^n, \P^m$ are closed subsets (not necessarily irreducible) then their product $X \times Y \subseteq \P^n \times \P^m$ is a closed subset, i.e. $\psi[X \times Y] \subseteq \psi[\P^n \times \P^m]$ is closed. This, combined with the actual statement of (b), will let us solve this problem.

        Indeed, as $X, Y$ are quasiprojective take some projective varieties $X', Y' \subseteq \P^n, \P^m$. Then by part (b), $X' \times Y' \subseteq \P^n \times \P^m$ is a projective variety, so we need only show that $X \times Y \subseteq X' \times Y'$ is an open subset. Indeed, we can decompose $X' \times Y' - X \times Y = (X' - X) \times Y' \cup X' \times (Y' - Y)$. Both parts of these union are closed as explained in the above paragraph. Hence, $X \times Y \subseteq X' \times Y'$ in the topology induced from $\psi$. In conclusion, $X \times Y$ is a quasiprojective variety.

        \item We want to show that $\psi[X \times Y] \subseteq \psi[\P^n \times \P^m]$ is an irreducible closed subset. We'll first show that if $X, Y$ are closed subsets of $\P^n, \P^m$ then $\psi[X \times Y] \subseteq \psi[\P^n \times \P^m]$ is closed. Indeed, recall that we had the map $\theta: k[z_{ij}] \longrightarrow k[x_0, \dots, x_n, y_0, \dots, y_m]$ via $z_{ij} |.x_i y_j$. Then the image of the Segre embedding $\psi: \P^n \times \P^m \longrightarrow \P^{(n+1)(m+1)-1}$ is given by $Z(\ker(\theta))$. This was proven in \ref{1.2.14}.

        So we need to find equations defining $\psi[X \times Y] \subseteq \P^{(n+1)(m+1)-1}$. As we're assuming $X, Y$ are closed in their respective projective spaces, we can write $X = Z(I)$ and $Y = Z(J)$ for homogeneous ideals $I, J \subseteq k[x_0, \dots, x_n], k[y_0, \dots, y_m]$. From now on, I may use the abbreviations $k[x] := k[x_0, \dots, x_n]$ and $k[y] = k[y_0, \dots, y_m]$. We have the quotient map $\overline{\phantom{a}}: k[x, y] \longrightarrow k[x, y]/(I \cup J)$. We consider then the composition $\overline \theta = \overline{\phantom a} \circ \theta: k[z] \longrightarrow k[x, y]/(I \cup J)$. We claim then that $Z(\ker(\overline \theta)) = X \times Y$.

        Before we show any containment, we need this expression to make sense. That is, we need $\ker(\overline \theta)$ to be homogeneous. This is true, as since $I, J$ are homogeneous we can write $(I \cup J) = (I^h \cup J^h)$, which is a homogeneous ideal in $k[x, y]$. Hence, $\theta$ and $\overline{\phantom{a}}$ are both maps of graded rings which respect the grading (although $\theta$ doubles grading). So $\overline \theta$ preserves grading and its kernel is therefore homogeneous.

        Now, we will show the equality $\psi[X \times Y] = Z(\ker \overline \theta)$.
        
        $``\subseteq"$ Let $(a, b) \in X \times Y$ and $f(z) \in \ker(\overline \theta)^h$. Then $f(\psi(a, b)) = \theta(f)(a, b)$. As $f \in \ker \overline \theta$, $\theta(f) \in \ker \overline{\phantom{a}} = (I^h \cup J^h)$. If we had any $g \in I^h$ then $g(a, b) = g(a) = 0$ as $a \in X = Z(I)$. Similarly, if $g \in J^h$ we'd have $g(a, b) = 0$. We can write $\theta(f)$ as a $k[x, y]$ linear combination of terms in $I^h \cup J^h$, so $\theta(f)(a, b) = 0$. Thus, $f(\psi(a, b)) = 0$ so $\psi(a, b) \in Z(\ker \overline \theta)$. In conclusion, $\psi[X \times Y] \subseteq Z(\ker \overline \theta)$.

        $``\supseteq"$ Let $P \in Z(\ker \overline \theta) \subseteq \ker \theta$. Then as $\psi$ is a bijection $\P^n \times \P^m \longrightarrow Z(\ker \theta)$, we can find unique $(a, b) \in \P^n \times \P^m$ such that $P = \psi(a, b)$. We seek to show that $(a, b) \in X \times Y$. Indeed, let's take some $f \in I^h$. If we can show that $f(a) = 0$ then we will have shown $a \in Z(I) = X$.
        
        All we know is that $P \in Z(\ker(\overline \theta))$, so we somehow want to pull $f$ back to $k[z]$. We have that some component $b_i$ of $b$ is nonempty. Then the polynomial $f(x_0 y_i, \dots, x_n y_i)$ is still homogeneous. Furthermore, by homogeneity of $f$, this is equal to $y_i^d f(x)$, where $d = \deg f$. Thus, $f(x_0 y_i, \dots, x_n y_i) \in (I \cup J)^h$. Let us write $f = \sum a_K x^K$. Then $f(x_0 y_i, \dots, x_n y_i) = \sum a_{k_0, \dots, k_n} (x_0 y_i)^{k_0} \cdots (x_n y_i)^{k_n}$. Hence, we take $g = \sum a_{k_0, \dots, k_n} z_{0i}^{k_0} \cdots z_{ni}^{k_n}$. This will be homogeneous (of degree $d$) and of course, $\theta(g) = f(x_0 y_i, \dots, x_n y_i) \in (I \cup J) = \ker \overline{\phantom{a}}$. Hence, $g \in (\ker \overline \theta)^h$.

        This therefore tells us that $g(P) = 0$. We have then that $0 = g(\psi(a, b)) = \theta(g)(a, b) = f(a_0 b_i, \dots, a_n b_i) = b_i^d f(a_0, \dots, a_n)$. As we chose $b_i \neq 0$, we must have $f(a) = 0$. Thus, $a \in Z(I) = X$. The same reasoning applies to show that $b \in Z(J) = Y$. Hence, $P \in \psi[X \times Y]$ and we have shown $Z(\ker \overline \theta) \subseteq \psi[X \times Y]$
                                                                                                   
        We have shown that $\psi[X \times Y] = Z(\ker \overline \theta)$, which is a closed subset of $\P^{(n + 1)(m + 1) - 1}$. Equivalently, $X \times Y \subseteq \P^n \times \P^m$ is closed in the structure induced by $\psi$. To conclude this, we need to show that if $X, Y$ are irreducible then their product $X \times Y$ is irreducible, which again really means that $\psi[X \times Y]$ is irreducible. As shown above, the equations defining $\psi[X \times Y]$ are given as $\ker(k[z] \longrightarrow k[x, y] \longrightarrow k[x, y]/(\p \cup \q))$, where $\p = I(X)$ and $\q = I(Y)$. As discussed in \ref{1.3.15}.a, this is isomorphic to $k[x]/\p \otimes_k k[y]/\q$. That this is a domain follows from \ref{1.3.15}.a. Alternatively, this can be shown directly using algebraic closure of $k$. This can be found in \cite[Ch. 5 \S 17.5 Cor. 3]{bourbaki-algebra2}. 

        \item Glue the affine open cover???

        Let $\pi_X, \pi_Y$ be the set theoretic projections from $X \times Y$. Let $Z = \psi[X \times Y]$. We could hope that the maps $p_X = \pi_X \circ \psi^{-1}$ and $p_Y = \pi_Y \circ \psi^{-1}$ are projections witnessing $Z$ as a product of the varieties $X, Y$. If we do indeed take this as our projections, then observe the isomorphisms
        \begin{align*}
            \mathsf{Set}(W, Z) &\cong \mathsf{Set}(W, X \times Y)\\
            &\cong \mathsf{Set}(W, X) \times \mathsf{Set}(W, Y)
        \end{align*}
        given by $f \mapsto \psi^{-1} \circ f \mapsto (\pi_X \circ \psi^{-1} \circ f, \pi_Y \circ \psi^{-1} \circ f) = (p_X \circ f, p_Y \circ f)$.

        So we need to show two things: that $p_X, p_Y$ are morphisms of varieties and that this isomorphism $f \mapsto (p_X \circ f, p_Y \circ f)$ restricts to $\mathsf{Var}(W, Z) \longrightarrow \mathsf{Var}(W, X) \times \mathsf{Var}(W, Y)$.

        I'll first show that $p_X$ and $p_Y$ are morphisms of varieties. Indeed, I'll explicitly describe a formula for it. Take some $P = [\{c_{ij}\}] \in Z$. Then $\psi^{-1}(P) = (a, b) \in X \times Y$ can be found as follows. Take some $(k, l)$ such that $c_{kl} \neq 0$. Then define $a_{i} = c_{il}$ and $b_j = c_{kj}$. That this actually works was done in \ref{1.2.4}. Hence, $p_X(P) = [c_{0l} : \dots : c_{nl}]$ and $p_Y(P) = [c_{k0} : \dots : c_{k m}]$. This looks like a morphism, but $(k, l)$ depend on $P$ so we are not done. What saves us is that the choice of $(k, l)$ can be taken locally constant in $P$, as if $c_{kl} \neq 0$ then we can take the same choice of $(k, l)$ throughout the open neighborhood $P \in Z - Z(z_{kl})$. Being a morphism is a local property, so $p_X$ and $p_Y$ are indeed morphisms.

        (actually also need to check well definedness, i.e. independence of choice of affine nbhd but this should follow from the relations)

        By the above computation, we already know that $p_X$, $p_Y$ give us a bijection $\mathsf{Set}(W, Z) \cong \mathsf{Set}(W, X) \times \mathsf{Set}(W, Y)$ for any variety $W$. We need this to restrict to $\mathsf{Var}$. That is, we want to say that $(p_x \circ \phi, p_Y \circ \phi)$ are morphisms of varieties if and only if $\phi$ is. We just showed that $p_X$ and $p_Y$ are morphisms of varieties, so one direction is immediate. We are now left to show that if $\phi: W \longrightarrow Z$ is a set map such that its projections $p_X \circ \phi$ and $p_Y \circ \phi$ are morphisms of varieties, then $\phi$ is a morphism of varieties.

        I'm not sure how this works honestly. The product diagram looks like this:
        $$
             \begin{tikzcd}
                 & W \arrow[dr, "\mathsf{Var}"] \arrow[dl, swap, "\mathsf{Var}"] \arrow[d, "\mathsf{Set}"]\\
                 Y & Z \arrow[l, "\mathsf{Var}"] \arrow[r, "\mathsf{Var}", swap] & X
             \end{tikzcd}
        $$
        so we sorta want a map of sheaves like this:
        $$
        \begin{tikzcd}
             & \mathcal O_W \\
             \mathcal O_Y \arrow[ur] \arrow[r] & \mathcal O_Z \arrow[u, dashed, "?"] & \mathcal O_X \arrow[ul] \arrow[l]
        \end{tikzcd}
        $$

        which somehow makes me want to say there's a tensor product thing happening but I really don't know.
    \end{enumerate}
\end{proof}
