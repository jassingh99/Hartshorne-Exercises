\label{1.2.17}

\textit{Complete intersections}. A variety $Y$ of dimension $r$ in $\P^n$ is a \textit{(strict) complete intersection} if $I(Y)$ can be generated by $n - r$ elements. $Y$ is a \textit{set-theoretic complete intersection} if Y can be written as the intersection of $n - r$ hypersurfaces.

\begin{enumerate}[label = (\alph*)]
    \item Let $Y$ be a variety in $\P^n$, let $Y = Z(\a)$; and suppose that $\a$ can be generated by $q$ elements. Then show that $\dim Y \geq n - q$.
    
    \item Show that a strict complete intersection is a set-theoretic complete intersection.
    
    \item The converse of (b) is false. For example let $Y$ be the twisted cubic curve in $\P^3$ (Ex. \ref{1.2.9}). Show that $I(Y)$ cannot be generated by two elements. On the other hand, find hypersurfaces $H_1, H_2$ of degrees $2, 3$ respectively, such that $Y = H_1 \cap H_2$.
    
    \item It is an unsolved problem whether every closed irreducible curve in $\P^3$ is a set-theoretic intersection of two surfaces. See Hartshorne [1] and Hartshorne [5, III, §5] for commentary.
    
    $\langle$\textit{My note}: These references are not to this textbook, but to the references in the textbook itself. For laziness, I do not include these in the bibliography.$\rangle$
\end{enumerate}

\begin{proof}
    \begin{enumerate}[label = (\alph*)]
        \item We want to apply Krull's principal ideal theorem here, but there's the issue that that is meant to hold for minimal primes, whereas here we are looking for homogeneous minimal primes over our homogeneous $\a$. So indeed, we first show the following lemma:
        \begin{lemma}
            Let $S$ be an $\omega$ graded ring. Then a minimal prime $\p$ of $S$ is homogeneous.
        \end{lemma}
        \begin{proof}
            To do this, we want to show that any prime $\p$ contains another prime $\q$ that is homogeneous. The natural choice is to take $\q = (\p^h)$, the ideal generated by the homogeneous elements of $\p$. This certainly tells us that $\q \subseteq \p$ and that $\q$ is homogeneous. Furthermore, $\p^h \subseteq \q^h \subseteq \p^h$ so we have $\q^h = \p^h$. Then to show that $\q$ is prime, we need only check this on homogeneous elements. Indeed, take $a, b \in S^h$ such that $ab \in \q^h = \p^h$. Then by primality of $\p$, $a \in \p$ or $b \in \p$. As these are homogeneous, one of these is therefore in $\p^h = \q^h \subseteq \q$. Thus, $\q$ is prime.

            Now if $\p$ was minimal then $\q \subseteq \p$ is an equality, and $\p = \q$ is homogeneous.
        \end{proof}
        Note that this proof also works for minimal primes over homogeneous ideals $I$, as we can apply the lemma to $S/I$, which is itself graded. Hence, any minimal prime over out given homogeneous ideal $\a$ must itself be homogeneous.

        Now, by Krull's principal ideal theorem, an minimal prime $\p \supseteq \a$ must have $\codim \p \leq q$. As discussed, $\p$ is homogeneous so we form the inclusion $Z(\p) \subseteq Z(\a) = Y$ of a maximal irreuducible susbet of $Y$. Then $\dim Y \geq \dim Z(\p)$. $\dim Z(\p) = \dim S(\p) - 1 = (n + 1) - \codim \p - 1 = n - \codim \p \geq n - q$.

        \item Let $I(Y) = (a_1, \dots, a_s)$ where $s = n - r$. Recall that $Y$ was assumed to be a variety so $I(Y)$ must be prime. Then for each $i$ there is some irreducible $f_i \mid a_i$ such that $f_i \in I(Y)$. Then $(f_1, \dots, f_s) \subseteq I(Y)$. On the other hand, $f_i \mid a_i$ so $a_i \in (f_i)$. Thus, $I(Y) = (a_1, \dots, a_s) \subseteq (f_1, \dots, f_s) \subseteq I(Y)$ so we achieve equality. Hence, we have $Y = \bigcap Z(f_i)$, which are all hypersurfaces.
        
        Of course, the problem insists that it is the intersection of $n - r$ hypersurfaces. The index of the intersection ranges from $1$ to $s = n - r$, but a priori there could be repeats in this intersections. However, the fact that $\dim Y = r$ precludes this. Indeed, suppose there were some repeats in the intersection, i.e. that some $Z(f_i) = Z(f_j)$. That would mean that we could generate $I(Y)$ by fewer than $s$ elements. But if $I(Y)$ is generated by $l$ many elements, part (a) above tells us that $r = \dim Y \geq n - l$. Thus, $l \geq n - r = s$, so $s$ is the minimal number of generators this ideal can have. Hence, it is the minimal number of hypersurfaces we need to intersect in $Y$.

        \item Take $Y = Z(xz - yz, yw - z^2, xw - yz)$. To write $Y = H_1 \cap H_2$ for hypersurfaces $H_i = Z(f_i)$ means that $\sqrt{(f_1, f_2)} = (xz - yz, yw - z^2, xw - yz)$. \href{https://en.wikipedia.org/wiki/Twisted_cubic}{Wikipedia} claims that we can take $f_1 = xz - y^2$ and $f_2 = z(yw - z^2) - w(xw - yz)$. One can then bash out this computation.

        On the other hand, we want to show that $I(Y)$ cannot be generated by $2$ elements. The na\"ive dimension approach using (a) won't work, as generating $I(Y)$ by two elements yields the inequality $\dim Y \geq 3 - 2$ which is correct. Hence, we'll approach this with Nakayama's lemma, noting that the given generators $a_1 = xz - yz$, $a_2 = yw - z^2$, $a_3 = xw - yz$ are linearly independent over $k$.

        Let's let $\p = I(Y)$. Now, we want a maximal ideal $\p \subseteq \m$ so that we can get a corresponding prime $\p S_\m$ in the local ring $S_\m$ on which we can apply Nakayama's lemma. Specifically, we note that generators of $\p S_\m$ correspond to generators of $\p S_\m / \p \m S_\m$ as a $S_\m / \m S_\m = k$ vector space, and in fact that minimal generating sets correspond to bases. We therefore want to compute the dimension of this latter space. Note that by flatness of localization, and the fact that $k = S/\m$ is a field, $\p S_\m / \p \m S_\m = \p / \p \m$.

        We therefore want to prove that $\dim_k \p / \p \m \geq 3$. Indeed, we will show that the $a_i \in \p$ are linearly independent in the quotient. Of course, we have no actually picked a maximal ideal $\m$ so we will take $\m = (x, y, z, w)$. Note that $0 \in Y$ so $\p \subseteq \m$. Now, let's try to understand the quotient. Take some $\sum f_i a_i \in \p$, where $f_i = \sum f_{iJ} X^J$ for $X = (x, y, z, w)$. We see that each term $X^J a_i \in \p \m$ for $J \neq (0, 0, 0, 0)$. Then in the quotient $\p / \p \m$, we have $\sum f_i a_i = \sum f_i(0) a_i$, so it is spanned as a $k$ vector space by the $a_i$. In other words, we have a well defined isomorphism $\p / m \p \longrightarrow \sum k a_i$ via $\sum f_i a_i \mapsto \sum f_i(0) a_i$ due to the isomorphism $S/\m \longrightarrow k$ via $f \mapsto f(0)$. Furthermore, the $a_i$ are $k$-linearly independent. Thus, $\dim_k \p / \p \m = 3$. If there was any smaller generating set for $\p$ then we could run through the same procedure and find a strictly smaller basis for $\p / \m \p$ - a contradiction.
    \end{enumerate}
\end{proof}
