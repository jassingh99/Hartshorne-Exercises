\label{1.2.17}

\textit{Complete intersections}. A variety $Y$ of dimension $r$ in $\P^n$ is a \textit{(strict) complete intersection} if $I(Y)$ can be generated by $n - r$ elements. $Y$ is a \textit{set-theoretic complete intersection} if Y can be written as the intersection of $n - r$ hypersurfaces.

\begin{enumerate}[label = (\alph*)]
    \item Let $Y$ be a variety in $\P^n$, let $Y = Z(\a)$; and suppose that $\a$ can be generated by $q$ elements. Then show that $\dim Y \geq n - q$.
    
    \item Show that a strict complete intersection is a set-theoretic complete intersection.
    
    \item The converse of (b) is false. For example let $Y$ be the twisted cubic curve in $\P^3$ (Ex. \ref{1.2.9}). Show that $I(Y)$ cannot be generated by two elements. On the other hand, find hypersurfaces $H_1, H_2$ of degrees $2, 3$ respectively, such that $Y = H_1 \cap H_2$.
    
    \item It is an unsolved problem whether every closed irreducible curve in $\P^3$ is a set-theoretic intersection of two surfaces. See Hartshorne [1] and Hartshorne [5, III, §5] for commentary.
    
    $\langle$\textit{My note}: These references are not to this textbook, but to the references in the textbook itself. For laziness, I do not include these in the bibliography.$\rangle$
\end{enumerate}

\begin{proof}
    
\end{proof}