\label{1.2.2}

For a homogeneous ideal $\a \subseteq S$, show that the following conditions are equivalent:

\begin{enumerate}[label = (\roman*)]
	\item $Z(\a) = \emptyset$;
	    
	\item $\sqrt{\a}$ =  either $S$ or the ideal $S_+ = \bigoplus_{d > 0} S_d$;
	    
	\item $\a \supseteq S_d$ for some $d > 0$.
\end{enumerate}

\begin{proof}
	Note that $S - S_+ = k - 0$. Hence, statement (ii) is equivalent to $\sqrt{a} \supseteq S_+$.
	
	$(iii) \implies (i).$ If $\a$ \supseteq S_d$ then $x_i^d \in \a$ for all $i$. Hence, $Z(\a)\subseteq \cap Z(x_i^d)=\emptyset$.
		
	$(i) \implies (ii).$ If $\deg f > 0$ then $f \in S_+$. Furthermore, it holds vacuously that for all $P \in Z(\a)=\emptyset$, $f(P)=0$. Hence, by the homogeneous Nullstellensatz, $f^n \in \a$ for some $n$. Hence, $\sqrt \a \supseteq S_+$. Thus, $\a=S_+$ or $\a=S$ as $S-S_+=k$. Note that we applied the homogeneous Nullstellensatz on $Z(\a)=\emptyset$. This is justified as $Z(\a)=\emptyset$ implies that $V(\a)-0=\emptyset$. Indeed, letting $\pi:\A^{n+1} \longrightarrow \P^n$, $\pi^{-1}[Z(\a)]=V(\a)-0$. With this formula in mind, we can see that the proof of \ref{1.2.1} is valid for the empty set.

	$(ii) \implies (iii).$ Let $\sqrt \a \supseteq S_+$. Then $x_i \in \sqrt \a$ for all $i$. Hence, $x_i^{d_i} \in \a$ for some $d_i$. Let $d=\sum d_i$. Now, let $x_0^{k_0} x_1^{k_1} \cdots x_n^{k_n}$ be a generic (modulo constants) degree $d$ monomial. That is, $\sum k_i=d$. If all $k_i > d_i$ then $d = \sum k_i > \sum d_i = d$. Hence, some $k_i \leq d_i$ so the term $x_i^{d_i}$ appears in this expression. Hence, $x_0^{k_)} x_1^{k_1} \cdots x_n^{d_n} \in \a$. Hence, $S_d \subseteq \a$.
\end{proof}
