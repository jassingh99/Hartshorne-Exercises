\label{1.2.6}

If $Y$ is a projective variety with homogeneous coordinate ring $S(Y)$, show that
$\dim S(Y) = \dim Y + 1$. [\textit{Hint}: Let $\varphi_i: U_i \longrightarrow \A^n$ be the homeomorphism of \cite[2.2]{hartshorne},
let $Y_i$ be the affine variety $\phi_i[Y \cap U_i]$, and let $A(Y_i)$ be its affine coordinate ring. Show that $A(Y_i)$ can be identified with the subring of elements of degree $0$ of the
localized ring $S(Y)_{x_i}$· Then show that $S(Y)_{x_i} \cong A(Y_i)[x_i, x_i^{-1}]$. Now use \cite[1.7]{hartshorne},
\cite[1.8A]{hartshorne}, and (Ex \ref{1.1.10}), and look at transcendence degrees. Conclude also that $\dim Y = \dim Y_i$ whenever $Y_i$ is nonempty.]

\begin{proof}
    We will follow this hint. We want to identity $A(Y_i)$ with $(S(Y)_{x_i})^0$, which is the set of all $\frac{f}{x_i^d}$ such that $f \in k[x_0, \dots, x_n]$ is homogeneous of degree $d$. We will consider the coordinate ring $A(Y_i)$ to have coordinates $x_0, \dots, \widehat{x_i}, \dots, x_n$. We define the map

    \begin{align*}
        k[x_0, \dots, \widehat{x_i}, \dots, x_n] &\longrightarrow (S(Y)_{x_i})^0\\
        f &\mapsto f\parens{\frac{x_0}{x_i}, \dots, \widehat{\frac{x_i}{x_i}}, \dots, \frac{x_n}{x_i}}
    \end{align*}

    It's not hard to see that this lands in the degree $0$ subring. We want this map to factor through $A(Y_i)$. What then is $J(Y_i)$? By definition, it is the $f \in k[a_0, \dots, \widehat{a_i}, \dots, a_n]$ such that $f(a) = 0$ for all $a \in Y_i$. Of course, $Y_i = \phi_i[Y \cap U_i]$ so any such $a$ will look like $\parens{\frac{a_0}{a_1}, \dots, \widehat{\frac{a_i}{a_i}}, \dots, \frac{a_n}{a_i}}$. Indeed, take some $f \in J(Y_i)$. Then $f$ is mapped to $\frac{\beta_i(f)}{a_i^d}$ where $\beta_i(f)$ is the homogenization of $f$ with respect to $a_i$, i.e. $\beta_i(f) = a_i^d f\parens{\frac{a_0}{a_i}, \dots, \widehat{\frac{a_i}{a_i}}, \dots, \frac{a_n}{a_i}}$. Here  we of course mean $d = \deg f$. Now, if $d = 0$ then $f$ is constant and vanishes on $Y_i$, so it must be $0$ (or $Y_i$ must be empty but this is a case we ignore). If $d \neq 0$ then consider some $[a_0 : \dots : a_n] \in Y$. If $a_i = 0$ then $\beta_i(f)([a_0 : \dots : a_n]) = a_i^d f\parens{\frac{a_0}{a_i}, \dots, \widehat{\frac{a_i}{a_i}}, \dots, \frac{a_n}{a_i}} = 0$ as $d > 0$. On the other hand, if $a_i \neq 0$ then the term $f\parens{\frac{a_0}{a_i}, \dots, \widehat{\frac{a_i}{a_i}}, \dots, \frac{a_n}{a_i}} = 0$ as $f \in J(Y_i)$. In any case, we see therefore that $\beta_i(f) \in I(Y)$. Hence, $\frac{\beta_i(f)}{x_i^d} \in I(Y)_{x_i}$. By flatness of localization, $S(Y)_{x_i} = k[x_0, \dots, x_n]_{x_i} / I(Y)_{x_i}$. Thus, we can quotient this map to yield $A(Y_i) \longrightarrow (S(Y)_{x_i})^0$. 

    We now seek to show that this map is an isomorphism, which we will do by exhibiting aninverse. Recall the ``dehomogenization" of $f$ with respect to $x_i$ is $\alpha_i(f) = f(x_0, \dots, 1, \dots, x_n)$ with $1$ in the $i^{th}$ position. We consider the map $(S(Y)_{x_i})^0 \longrightarrow A(Y_i)$ via $\frac{f}{x_i^d} \mapsto \alpha_i(f)$, where of course $f$ is taken to be homogeneous and degree $d$. In other words, this map evaluates a rational function at $(x_0, \dots, 1, \dots, x_n)$.

    We first have to show that this map is actually well defined on these quotients. Indeed, let's begin with some $f \in I(Y)_{x_i}$ of homogeneous degree $0$, so $f = \frac{g}{x_i^d}$ with $g \in I(Y)^h$ of degree $d$. Then this maps to $g(x_0, \dots, 1, \dots, x_n)$. Take some $\parens{\frac{a_0}{a_1}, \dots, \widehat{\frac{a_i}{a_i}}, \dots, \frac{a_n}{a_i}} \in Y_i$. Then $g(x_0, \dots, 1, \dots, x_n)\parens{\frac{a_0}{a_1}, \dots, \widehat{\frac{a_i}{a_i}}, \dots, \frac{a_n}{a_i}}\right = g\parens{\frac{a_0}{a_i}, \dots, 1, \dots, \frac{a_n}{a_i}} = 0$ as $g \in I(Y)^h$. Thus, this does descend to a well defined map on the quotient $(S(Y)_{x_i})^0$.

    It is now easy enough to verify that these maps are inverse. Let $f \in A(Y_i)$. Then $f \mapsto \frac{\beta_i(f)}{x_i^d} \mapsto \alpha_i(\beta_i(f))$. This is $f\parens{\frac{x_0}{x_i}, \dots, \widehat{\frac{x_i}{x_i}}, \dots, \frac{x_n}{x_i}}(x_0, \dots, 1, \dots, x_n) = f$. On the other hand, let $\frac{f}{x_i^d} \in (S(Y)_{x_i})^0$. Then $\frac{f}{x_i^d} \mapsto \alpha_i(f) \mapsto \frac{\beta_i(\alpha_i(f))}{x_i^d}$. The numerator here is $\beta_i(f(x_0, \dots, 1, \dots, x_n)) = x_i^d f\parens{\frac{x_0}{x_i}, \dots, \frac{x_i}{x_i}, \dots, \frac{x_n}{x_i}} = f$ as $f$ is homogeneous of degree $f$. 

    In summary, we have defined the following maps:
    $$
    \begin{tikzcd}[ampersand replacement = \&, row sep = 0]
        x_j \arrow[r, mapsto] \& \frac{x_j}{x_i} \\
        A(Y_i) \arrow[r, leftrightarrow, "\sim"] \& (S(Y)_{x_i})^0 \\
        \begin{cases}
            1 & i = j\\
            x_j & i \neq j
        \end{cases} \& x_j \arrow[l, mapsto] 
    \end{tikzcd}
    $$

    Next, we will extend this to a map $A(Y_i)[x_i, x_i^{-1}] \longrightarrow S(Y)_{x_i}$ via sending $x_i \mapsto x_i$. The image of this map contains $x_i$, $x_i^{-1}$ and $(S(Y)_{x_i})^0$. Given $f$ homogeneous, $\frac{f}{x_i^d} \in (S(Y)_{x_i})^0$, so $f \in (S(Y)_{x_i})^0[x_i]$. Hence, this map is onto. By flatness, it suffices to show that the map $A(Y_i)[x_i] \longrightarrow S(Y)_{x_i}$ is injective.

    Recall that each $\phi(f_k)$ has degree $0$ in $S(Y)_{x_i}$. Then $\phi(f_k) x_i^k$ has degree $k$ in this ring, so each term in $\sum \phi(f_k) x^k$ has different degree, so these are necessarily linearly independent. Hence, for this sum to vanish, each term $\phi(f_k) x_i^k$ must vanish as well. Since we are in a domain ($Y$ is a variety), this means that each $\phi(f_k) = 0$. $\phi$ has already been shown to be an isomorphism $A(Y_i) \longrightarrow (S(Y)_{x_i})^0$, so each $f_k$ is therefore $0$. Thus, the extension of $\phi$ to $A(Y_i)[x_i, x_i^{-1}] \longrightarrow S(Y)_{x_i}$ is an isomorphism as well.

    Now, to compute $\dim S(Y)$ we need to compute $\dim A(Y_i)[x_i, x_i^{-1}]$, which is equal to the transcendence degree of $qf(A(Y_i)[x_i]) = qf(A(Y_i))(x_i)$. $x_i$ was a formal variable for $A(Y_i)$, as we took $A$ to have coordinates $x_0, \dots, \widehat{x_i}, \dots, x_n$. Hence, $\trdeg_k qf(A(Y_i))(x_i) = \trdeg_k qf(A(Y_i)) + 1 = \dim A(Y_i) + 1$. By \ref{1.1.10}(b), $\dim Y = \sup \dim Y_i$. $Y_i$ is empty when $Y \cap U_i = \emptyset$, i.e. when $Y \subseteq Z(x_i)$. In that case, $x_i \in I(Y)$ so $x_i = 0$ in $S(Y)_{x_i}$. When this is not the case (and it must not be the case for some $x_i$), then $x_i \neq 0$ in $S(Y)$ so we have inclusions $S(Y) \subseteq S(Y)_{x_i} \subseteq qf(S(Y))$. We have computed that $qf(S(Y)_{x_i}) = qf(A(Y_i))(x_i)$, which has transcendence degree $\dim A(Y_i) + 1$, which is $\dim Y_i + 1$. Thus, for all $i$ such that $Y_i \neq \emptyset$, we have $\dim Y_i = \dim S(Y) - 1$. Taking sups yields $\dim Y = \dim S(Y) - 1$, and that $\dim Y = \dim Y_i$ whenever $Y_i \neq \emptyset$.
\end{proof}                                                                              
