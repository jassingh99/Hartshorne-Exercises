\label{1.2.1}

Prove the ``homogeneous Nullstellensatz," which says if $\a \subseteq S$ is a homogeneous ideal, and if $f \in S$ is a homogeneous polynomial with $\deg f > 0$, such that $f(P) = 0$ for all $P \in Z(\a)$ in $\P^n$, then $f^q \in \a$ for some $q > 0$. [\textit{Hint}: Interpret the problem in terms of the affine $(n + 1)$-space whose affine coordinate ring is $S$, and use the
usual Nullstellensatz, \cite[1.3A]{hartshorne}.

\begin{proof}
	$\A^{n + 1}$ has coordinate ring $S$. For notational ease, I will denote $V(\a) \subseteq \A^{n + 1}$ to be the affine variety defined by $\a$, as opposed to $Z(\a)$ which refers heretofor only to the projective variety. The given condition on $f$ is that for all $P \in Z(\a)$, $f(P) = 0$. By definition, $f(P) = 0$ means that $f(a_0, \dots, a_n) = 0$ for any homogeneous coordinates $(a_0, \dots, a_n)$ of $P$. Take now some $(a_0, \dots, a_n) \in V(\a)$. We claim that $f(a_0, \dots, a_n) = 0$. As $f$ is homogeneous, $f(0) = 0$ so suppose $(a_0, \dots, a_n)$ is nonzero and let $P \in \P^n$ be the point it represents. Let $g \in \a$ be homogeneous. Then $g(a_0, \dots, a_n) = 0$ so $g(P) =  0$ and $P \in Z(\a)$. Hence, $f(P) = 0$ so by definition, $f(a_0, \dots, a_n) = 0$. Thus, $f \in I(V(\a)) = \sqrt{\a}$ by the Nullstellensatz.
\end{proof}
