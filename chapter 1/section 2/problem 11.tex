\label{1.2.11}

\textit{Linear Varieties in} $\P^n$. A hypersurface defined by a linear polynomial is called a \textit{hyperplane}.

\begin{enumerate}[label = (\alph*)]
    \item Show that the following two conditions are equivalent for a variety $Y$ in $\P^n$:
    \begin{enumerate}[label = (\roman*)]
        \item $I(Y)$ can be generated by linear polynomials.
        \item $Y$ can be written as an intersection of hyperplanes.
    \end{enumerate}
    In this case we say that $Y$ is a linear variety in $\P^n$.
    
    \item If $Y$ is a linear variety of dimension $r$ in $\P^n$, show that $I(Y)$ is minimally generated by $n - r$ linear polynomials.
    
    \item Let $Y,Z$ be linear varieties in $\P^n$, with $\dim Y = r$, $\dim Z = s$. If $r + s - n \geq 0$, then $Y \cap Z \neq \emptyset$. Furthermore, if $Y \cap Z \neq \emptyset$, then $Y\cap Z$ is a linear variety of dimension $\geq r + s - n$. (Think of $\A^{n + 1}$ as a vector space over $k$, and work with its subspaces.)
\end{enumerate}

\begin{proof}
    \begin{enumerate}[label=(\alph*)]
        \item $(i \Longrightarrow ii)$. Let $I(Y) = (f_1, \dots, f_r)$ linear. Then $Y = Z(I(Y)) = \bigcap Z(f_i)$.

        $(ii \Longrightarrow i)$. Let $Y = \bigcap_{i \in I} Z(f_i)$. We therefore have $I(Y) = \sqrt{(f_i : i \in I)}$. As $S$ is N\"otherian we can find a finite set of generators among the $f_i$, i.e. $(f_1, \dots, f_r) = (f_i : i \in I)$. Indeed, find the maximal such finitely generated sub-ideal. Thus, $Y = Z(f_1, \dots, f_r)$. Write $f_i = \sum f_{ij} x_j$. Then $[a_0 : \dots : a_n] \in Y$ iff $(f_{ij})} \begin{pmatrix}
            a_0\\
            \vdots\\
            a_n
        \end{pmatrix} = 0$. The kernel of the matrix $(f_{ij})$ will have some basis $v_1, \dots, v_s$ so take a matrix $A$ which sends $e_i \mapsto v_i$ and the rest to whatever basis you extend it to. Then $(f_{ij}) A P = 0$ precisely when $P = (*, \dots, *, 0, \dots, 0)$ with $s$ many $*'s$. Thus $A$ represents a linear change of coordinates on $\P^n$ which sends $Z(f_1, \dots, f_r)$ to $Z(y_{s + 1}, \dots, y_n)$, whose ideal is of course just $(y_{s + 1}, \dots, y_n)$. This leads us to the general fact that a proper ideal generated by finitely many linear polynomials is necessarily prime. If you don't like this, take the transformation on the polynomials themselves.

        \item Suppose that $I(Y) = (f_1, \dots, f_s)$ linear polynomials. Then by Krull's principal ideal theorem, $\codim I(Y) \leq s$. As $I(Y)$ is prime, (see (a) or the assumption that $Y$ is a variety) we compute $\dim S(Y) = (n + 1) - \codim I(Y) \geq n + 1 - s$. Of course, by \ref{1.2.6}, $\dim S(Y) = \dim Y + 1 = r + 1$. Hence, we must have $r + 1 \geq n + 1 - s$ and therefore that $s \geq n - r$.

        \item Recall from \ref{1.2.10} the map $\theta: \A^{n + 1} - 0 \longrightarrow \P^n$ sending $(a_0, \dots, a_n) \mapsto [a_0 : \dots : a_n]$, and the cone $C(Y) = \theta^{-1}[Y] \cup \curly{0}$ for a closed subset $Y \subseteq \P^n$. Then as $\dim Y = r$ and $\dim Z = s$ we have $\dim C(Y) = r + 1$ and $\dim C(Z) = s + 1$. The condition that $r + s - n \geq 0$ is equivalent to $(r + 1) + (s + 1) - (n + 1) \geq 1$, i.e. that $\dim C(Y) + \dim C(Z) - \dim A^{n + 1} \geq 1$. The idea then is to use the fact that a line in $A^{n+1}$ is precisely a point in $\P^n$.
        
        Recall also that $C(Z(I)) = V(I) \subseteq \A^{n + 1}$. As $Y, Z$ are linear they are defined by the zero set of finitely many homogeneous linear polynomials. Hence, their affine cones are the affine zero sets of those same polynomials, and are therefore subspaces of the vector space $\A^{n + 1} = k^{n + 1}$. Hence, from the second isomorphism theorem, we observe that $\dim C(Y) + \dim C(Z) - \dim A^{n + 1} \leq \dim (C(Y) \cap C(Z))$.

        To sum the above up, $r + s - n \geq 0$ iff $\dim (C(Y) \cap C(Z)) \geq 1$. In other words, that there is some nonzero $v \in \dim (C(Y) \cap C(Z))$ Then the line $\theta(v) \in Y \cap Z$ witnesses the fact that this is nonempty. Long story short, $Y \cap Z \neq \emptyset$ iff their cones share a common line.
    \end{enumerate}
\end{proof}
