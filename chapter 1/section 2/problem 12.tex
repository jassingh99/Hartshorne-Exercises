\label{1.2.12}

\textit{The $d-Uple$ Embedding}. For given $n, d > 0$, let $M_0, M_1, \dots ,M_N$ be all the monomials of degree $d$ in the $n + 1$ variables $x_0 , \dots, x_n$, where $N = \binom{n + d}{n} - 1$. We
define a mapping $\rho_d: \P^n \longrightarrow \P^N$ by sending the point $P = (a_0, \dots ,a_n)$ to the point $\rho_d(P) = (M_0(a), \dots ,M_N(a))$ obtained by substituting the $a_i$ in the monomials $M_j$. This is called the $d$-uple \textit{embedding} of $\P^n$ in $\P^N$. For example, if $n = 1$, $d = 2$, then
$N = 2$, and the image $Y$ of the 2-uple embedding of $\P^1$ in $\P^2$ is a conic.

\begin{enumerate}[label = (\alph*)]
    \item Let $\theta: k[y_0 , \dots, y_N] \longrightarrow k[x_0 , \dots, x_n]$ be the homomorphism defined by sending $y_i$ to $M_i$, and let $\a$ be the kernel of $\theta$. Then $\a$ is a homogeneous prime ideal, and so $Z(\a)$ is a projective variety in $\P^N$.
    
    \item Show that the image of $\rho_d$ is exactly $Z(\a)$. (One inclusion is easy. The other will require some calculation.)
    
    \item Now show that $\rho_d$ is a homeomorphism of $\P^n$ onto the projective variety $Z(\a)$.
    
    \item Show that the twisted cubic curve in $\P^3$ (Ex. \ref{1.2.9}) is equal to the 3-uple embedding of $\P^1$ in $\P^3$, for suitable choice of coordinates.
\end{enumerate}

\begin{proof}
    \begin{enumerate}[label = (\alph*)]
        \item Proving the homogeneity of $\a$ comes from $\theta$ being defined as evaluation by homogeneous polynomials. Indeed, let $f \in \a$, so that $\theta(f) = f(M_1, \dots, M_N) = 0$. Let $f = \sum f_e$ be the homogeneous decomposition of $f$. Then $\sum f_e(M_1, \dots, M_N) = 0$. Furthermore, as each $M_i$ is homogeneous of degree $d$, the degree of $f_e(M_1, \dots, M_N)$ is $de$. As $d \geq 1$, each $f(M_1, \dots, M_N)$ has a distinct degree. Thus, by linear independence, each $f_e(M_1, \dots, M_N) = 0$. Hence, they are all in $\a$ so $\a$ is homogeneous. Primality is immediate from the fact that $k[x_0, \dots, x_n]$ is a domain.

        \item For the easier inclusion, take some $\rho_d([a_0 : \dots : a_n]) = [M_0(a) : \dots : M_N(a)]$ and let $f \in Z(\a)^h$. Then $f(M_0(a), \dots, M_N(a)) = \theta(f)(a) = 0$ so $\im \rho_d \subseteq Z(\a)$. 

        For the converse JustDoIt or possibly seek \href{https://www.mathreference.com/ag-pv,duple.html}{this reference} if I resign.

        We will define an inverse map $\psi: Z(\a) \longrightarrow \P^n$. 

        \item whatever it should work hopefully. We'd expect that $\rho_d^{-1}[Z(I)] = Z(\theta[I])$ and that $\rho_d[Z(J)] = Z(\theta^{-1}[J^d])$, where by $J^d$ I mean to take your favorite homogeneous generators $J = (S)$ and let $S^d$ be raising everthing in $S$ to the $d^{th}$ power so that it is in the image of $\theta$.

        \item We did this is \ref{1.2.9} already.
    \end{enumerate}
\end{proof}
