\label{1.2.12}

\textit{The $d-Uple$ Embedding}. For given $n, d > 0$, let $M_0, M_1, \dots ,M_N$ be all the monomials of degree $d$ in the $n + 1$ variables $x_0 , \dots, x_n$, where $N = \binom{n + d}{n} - 1$. We
define a mapping $\rho_d: \P^n \longrightarrow \P^N$ by sending the point $P = (a_0, \dots ,a_n)$ to the point $\rho_d(P) = (M_0(a), \dots ,M_N(a))$ obtained by substituting the $a_i$ in the monomials $M_j$. This is called the $d$-uple \textit{embedding} of $\P^n$ in $\P^N$. For example, if $n = 1$, $d = 2$, then
$N = 2$, and the image $Y$ of the 2-uple embedding of $\P^1$ in $\P^2$ is a conic.

\begin{enumerate}[label = (\alph*)]
    \item Let $\theta: k[y_0 , \dots, y_N] \longrightarrow k[x_0 , \dots, x_n]$ be the homomorphism defined by sending $y_i$ to $M_i$, and let $\a$ be the kernel of $\theta$. Then $\a$ is a homogeneous prime ideal, and so $Z(\a)$ is a projective variety in $\P^N$.
    
    \item Show that the image of $\rho_d$ is exactly $Z(\a)$. (One inclusion is easy. The other will require some calculation.)
    
    \item Now show that $\rho_d$ is a homeomorphism of $\P^n$ onto the projective variety $Z(\a)$.
    
    \item Show that the twisted cubic curve in $\P^3$ (Ex. \ref{1.2.9}) is equal to the 3-uple embedding of $\P^1$ in $\P^3$, for suitable choice of coordinates.
\end{enumerate}

\begin{proof}
    \begin{enumerate}[label = (\alph*)]
        \item Proving the homogeneity of $\a$ comes from $\theta$ being defined as evaluation by homogeneous polynomials. Indeed, let $f \in \a$, so that $\theta(f) = f(M_1, \dots, M_N) = 0$. Let $f = \sum f_e$ be the homogeneous decomposition of $f$. Then $\sum f_e(M_1, \dots, M_N) = 0$. Furthermore, as each $M_i$ is homogeneous of degree $d$, the degree of $f_e(M_1, \dots, M_N)$ is $de$. As $d \geq 1$, each $f(M_1, \dots, M_N)$ has a distinct degree. Thus, by linear independence, each $f_e(M_1, \dots, M_N) = 0$. Hence, they are all in $\a$ so $\a$ is homogeneous. Primality is immediate from the fact that $k[x_0, \dots, x_n]$ is a domain.

        \item For the easier inclusion, take some $\rho_d([a_0 : \dots : a_n]) = [M_0(a) : \dots : M_N(a)]$ and let $f \in Z(\a)^h$. Then $f(M_0(a), \dots, M_N(a)) = \theta(f)(a) = 0$ so $\im \rho_d \subseteq Z(\a)$. 

        We will define an inverse map $\psi: Z(\a) \longrightarrow \P^n$. For a multi-index $I$ we will define $I_j = (i_0, \dots, i_j + 1, \dots, i_n)$. If we take $\sum I = d - 1$ then we will define a map $\psi_I: Z(\a) \longrightarrow \P^n$ via $[\{a_J\}] \mapsto [a_{I_0} : \dots : a_{I_n}]$. Essentially, we are extracting $n + 1$ coordinates from $[\{a_J\}]$, starting from $I_0$ and proceeding in lexicographic order. Essentially, we seek to show that this, with all the relations of $\a$, are all we need to recover $[\curly{a_I}]$.

        First, we will show that $\psi_I$ is independent of the choice of $I$. Indeed, take another multi-index $K$ with $\sum K = d - 1$. Then to compare $\psi_I([\curly{a_J}])$ and $\psi_K([\curly{a_J}])$ we need to compare the ratios $\frac{a_{I_j}}{a_{K_j}}$. They define the same point in projective space iff all these ratios are the same (ignoring division by $0$). Observe that $I_j + K_0 = I_0 + K_j$, so $\a$ contains the polynomial $y_{I_j} y_{K_0} - y_{I_0} y_{K_j}$. As we are defining our $\psi$ on $Z(\a)$, we therefore have $a_{I_j} a_{K_0} = a_{I_0} a_{K_j}$, i.e. that $\frac{a_{I_j}}{a_{K_j}} = \frac{a_{I_0}}{a_{K_0}}$ for all $J$. Hence, $\psi_I = \psi_K$ so we can simply call it $\psi$.

        Start with some $a = [a_0 : \dots : a_n] \in \P^n$. Take some $a_i \neq 0$ and take $I = (0, \dots, 0, d - 1, 0, \dots, 0)$ in the $i^{th}$ position. $\rho_d(a) = [\curly{a^J}]$, and applying $\psi = \psi_I$ to this yields $[a^{I_0} : \dots : a^{I_n}]$. $a^{I_j} = a^{(0, \dots, 0, d - 1, 0, \dots, 0) + (0, \dots, 0, 1, 0, \dots, 0)} = a_i^{d - 1} a_j$. Hence, $\psi(\rho_d(a)) = [a_i^{d} : a_i^{d - 1} a_1 : \dots : a_i^{d - 1} a_n] = a$.

        On the other hand, take some $[\curly{a_J}] \in Z(\a)$. There is some index $I_0$ for which $a_{I_0} \neq 0$ (ok fine maaayybbeee all of the $(0, *, \dots, *)$ terms vanish but I'm sure this won't work if I bash some relations so whatever). Applying $\psi = \psi_I$ yields $[a_{I_0} : \dots : a_{I_n}]$. If we apply $\rho_d$ to this we end up with $\bracket{\prod_i a_{I_i}^{j_i}}$ ranging over all $J = (j_0, \dots, j_n)$. We want this to equal our original $[\curly{a_J}]$,so again we consider the ratios $\parens{a_{I_i}^{j_i} / a_J}$ vs $\parens{a_{I_i}^{k_i} / a_K}$. In other words, we seek to show that $a_K \prod a_{I_i}^{j_i} = a_J \prod a_{I_i}^{k_i}$. Indeed, this corresponds to the polynomial $y_K \prod y_{I_i}^{j_i} - y_J \prod y_{I_i}^{k_i}$. Note that on the left hand side, the ``signature" is $K + \sum j_i I_i$ and on the right hand side it is $J + \sum k_i I_i$. We'll let $\hat{i} = (0, \dots, 0, 1, 0, \dots, 0)$. Then $J + \sum k_i I_i = J + \sum k_i (I + \hat{i})$. This is $J + \sum k_i I + \sum k_i \hat{i} = J + K + dI$. The same computation works for the other term, so the signatures agree and therefore $y_K \prod y_{I_i}^{j_i} - y_J \prod y_{I_i}^{k_i} \in \a$. Thus, $[\curly{a_J}]$ vanishes on this so we get the desired equality $a_K \prod a_{I_i}^{j_i} = a_J \prod a_{I_i}^{k_i}$ so $\rho_d(\psi([\curly{a_J}])) = [\curly{a_J}]$.

        We have therefore proven that these two maps are inverse. In fact, we could have done this for just $Z(\curly{\prod y_{K_i} - \prod y_{J_i} : \sum K_i = \sum J_i})$, so these generate $\a$. If we try hard enough we may even show that we only need the degree two examples, but whatever.

        \item We'd expect that $\rho_d^{-1}[Z(I)] = Z(\theta[I])$ and that $\rho_d[Z(J)] = Z(\theta^{-1}[J^d])$, where by $J^d$ I mean to take your favorite homogeneous generators $J = (S)$ and let $S^d$ be raising everthing in $S$ to the $d^{th}$ power so that it is in the image of $\theta$. It seems pretty believable that $\psi$ and $\rho_d$ are continuous though, as they're all just evaluation at a bunch of homogeneous polynomials of the same degree so they better be continuous. In fact, see \ref{1.3.4} to see that they are morphisms of varieties, hence continuous. No need to worry about circularity, we never used continuity in the proof of \ref{1.3.4}, only that these were inverse.

        \item We did this is \ref{1.2.9} already.
    \end{enumerate}
\end{proof}
