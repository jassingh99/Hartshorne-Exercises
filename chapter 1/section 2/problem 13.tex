\label{1.2.13}

Let $Y$ be the image of the 2-uple embedding of $\P^2$ in $\P^5$. This is the \textit{Veronese surface}. If $Z \subseteq Y$ is a closed curve (a \textit{curve} is a variety of dimension $1$), show that
there exists a hypersurface $V \subseteq \P^5$ such that $V \cap Y = Z$.

\begin{proof}
    Take indeed some closed curve $Z \subseteq Y$, i.e. a variety of dimension $1$. By \ref{1.2.12} we have a homeomorphism $\rho_2: \P^2 \longrightarrow Y = Z(\a)$ where $\a$ is defined as in \ref{1.2.12}. Let $\Gamma = \rho_2^{-1}[Z]$. This is a codimension $1$ variety so it equals $Z(f)$ for some $f \in k[x_0, x_1, x_2]$ irreducible and homogeneous (\ref{1.2.8}). To find a hypersurface $V \subseteq \P^5$ we therefore want some $g \in k[y_0, \dots, y_5]$ homogeneous and irreducible such that $V = Z(g)$. The intersection $Y \cap V = Z(\a) \cap Z(g) = Z(\a + (g))$. We therefore want the image of $\rho_2$ to be $Z(\a + (g))$. On the coordinate rings, this would mean we have $\theta: k[y_0, \dots, y_5] / \sqrt{(\a + (g))} \longrightarrow k[x_0, x_1, x_2]/(f)$ an isomorphism. Then we essentially need $\theta[\sqrt{(\a + (g))}] = (f)$. As $\a$ is the kernel we're looking for $\sqrt{(\theta(g))} = (f)$

    Let's first consider the image of $\theta$ in this case. Now, $\theta(f) = f(x_0^2, x_0 x_1, x_0 x_2, x_1^2, x_1 x_2, x_2^2)$ so certainly anything in the image has all homogeneous summands of even degree. On the other hand, consider an even degree monomial $x_0^i x_1^j x_2^k$, i.e. $i + j + k \equiv 0 \pmod{2}$ If these are all $0$ then we are done via $y_{200}^{i/2} y_{020}^{j/2} y_{002}^{k/2}$. Alternatively, we could have one of these $0$ and the other two are $1$. For instance, if $i, j \equiv 1 \pmod{2}$ then we could take $y_{200}^{\frac{i-1}{2}} y_{020}^{\frac{j - 1}{2}} y_{110} y_{002}^{k/2}$. Therefore we have that $\im \theta$ is the set of polynomials whose homogeneous components are all even degree. And in fact, for a homogeneous even degree guy, we can find a preimage which is also homogeneous. Thus, we can solve $\sqrt{(\theta(g))} = (f)$ by taking $\theta(g) = f^2$ and $g$ homogeneous.

    We now need to find an irreducible $g$ satisfying the above. Suppose we were lucky and found a homogeneous $g$ such that $\theta(g) = f$. If $g = ab$ then $f = \theta(a) \theta(b)$, so WLOG take $\theta(a) \in k^\times$, as $f$ is irreducible. Then as $\theta$ sends each $y$ to a homogeneous degree monomial, only constants can map to constants. Hence, $a \in k^\times$ so $g$ is irreducible. On the other hand, if we only have $\theta(g) = f^2$ then again write $g = ab$ whence $f^2 = \theta(a) \theta(b)$. If $g$ was already irreducible then we'd of course be done, so suppose not. Then $\theta(a) = \theta(b) = f$ as $k[x_0, x_1, x_2]$ is a UFD. As $ab = g$ homogeneous, $a, b$ are homogeneous. Then by the previous case, they are irreducible.

    In any case, we have shown that we can find an irreducible homogeneous $g \in k[\{y_I\}]$ such that $\theta(g) = f$ or $f^2$. As discussed, we therefore take $V = Z(g)$ as our candidate hypersurface whose ``shadow" onto the $d$-uple embedded $\P^2$ is our fixed curve $Z$. That is, we want to show that $V \cap Y = Z(\a + (g)) = Z = \rho_2[Z(f)]$. Of course, we already know $Z \subseteq Y$. Take some $P \in Z$, which we know is of the form $\rho_2(Q)$ for $Q \in Z(f)$. Then $g(P) = g(\rho_2(Q)) = \theta(g)(Q) = f(Q) = 0$ (or $f^2(Q)$ but kjwe;lkjsDLf). Hence, we also have $Z \subseteq Z(g)$ so $Z \subseteq Y \cap V$. On the other hand, take some $P \in Y \cap V = Z(\a + (g))$. Then $P \in Y$ so write $P = \rho_2(Q)$. We want to show that $Q \in Z(f)$. Indeed, let $g(P) = g(\rho_2(Q)) = \theta(g)(Q) = 0$. Of course, $\theta(g) = f^{1.5 \pm 0.5}$ so $f(Q) = 0$ and $P \in \rho_2[Z(f)] = Z$ as desired. Thus, we have $V \cap Y \subseteq Z \subseteq V \cap Y$.
\end{proof}
