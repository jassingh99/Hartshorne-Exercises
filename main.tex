
\documentclass[12pt]{article}
 
\usepackage[margin=1in]{geometry}
\usepackage[new]{old-arrows}
\usepackage{amsmath,enumitem,amsthm,amssymb,graphicx,mathtools,tikz,hyperref,yfonts,tikz-cd, subfiles, datetime, adjustbox, extarrows,old-arrows}
\usepackage[nottoc]{tocbibind}

\usepackage[backend=biber,style=alphabetic,sorting=ynt]{biblatex}
\addbibresource{bob.bib}
\DeclareFieldFormat{postnote}{#1}
\DeclareFieldFormat{multipostnote}{#1}

\usetikzlibrary{positioning,lindenmayersystems}

\usepackage{graphicx}
\graphicspath{ {./images/} }
\usepackage[rightcaption]{sidecap}
\usepackage{wrapfig}

\newdateformat{monthdayyeardate}{%
\monthname[\THEMONTH] \THEDAY, \THEYEAR}

\setcounter{tocdepth}{2}

\pagestyle{headings}

\usepackage{hyperref}
\hypersetup{
    colorlinks  = true,
    linkcolor   = blue,
    filecolor   = magenta,      
    urlcolor    = cyan,
}

\newtheorem{theorem}{Theorem}[section]
\newtheorem{corollary}{Corollary}[theorem]
\newtheorem{lemma}[theorem]{Lemma}

\theoremstyle{definition}
\newtheorem{definition}{Definition}[section]

\theoremstyle{definition}
\newtheorem*{example}{Example}

\theoremstyle{definition}
\newtheorem*{examples}{Examples}

\theoremstyle{definition}
\newtheorem*{remark}{Remark}

\theoremstyle{definition}
\newtheorem*{remarks}{Remarks}

\theoremstyle{definition}
\newtheorem*{conventions}{Conventions}

\DeclareMathOperator{\spec}{Spec}
\DeclareMathOperator{\Frac}{Frac}
\DeclareMathOperator{\nil}{nil}
\DeclareMathOperator{\rad}{rad}
\DeclareMathOperator{\ann}{Ann}
\DeclareMathOperator{\cok}{cok}
\DeclareMathOperator{\im}{im}
\DeclareMathOperator{\Hom}{Hom}
\DeclareMathOperator{\id}{id}
\DeclareMathOperator{\tor}{Tor}
\DeclareMathOperator{\ext}{Ext}
\DeclareMathOperator{\codim}{codim}
\DeclareMathOperator{\trdeg}{tr deg}
\DeclareMathOperator{\tot}{Tot}
\DeclareMathOperator{\qf}{qf}
\DeclareMathOperator{\mor}{Mor}
\DeclareMathOperator*{\colim}{colim}
\newcommand{\R}{\mathbb R}
\newcommand{\C}{\mathbb C}
\newcommand{\Z}{\mathbb Z}
\newcommand{\N}{\mathbb N}
\newcommand{\Q}{\mathbb Q}
\renewcommand{\P}{\mathbb P}
\newcommand{\A}{\mathbb A}
\newcommand{\F}{\mathbb F}
\newcommand{\p}{\mathfrak p}
\newcommand{\q}{\mathfrak q}
\newcommand{\m}{\mathfrak m}
\renewcommand{\a}{\mathfrak a}
\renewcommand{\b}{\mathfrak b}
\newcommand{\rp}{\R\P}
\newcommand{\cp}{\C\P}
\newcommand{\transverse}{\mathrel{\text{\tpitchfork}}}
\makeatletter
\newcommand{\tpitchfork}{%
  \vbox{
    \baselineskip\z@skip
    \lineskip-.52ex
    \lineskiplimit\maxdimen
    \m@th
    \ialign{##\crcr\hidewidth\smash{$-$}\hidewidth\crcr$\pitchfork$\crcr}
  }%
}
\makeatother
\newcommand{\diff}[2]{\frac{\partial{#1}}{\partial{#2}}}
\newcommand{\parens}[1]{{\left(#1\right)}}
\newcommand{\bracket}[1]{{\left[#1\right]}}
\newcommand{\curly}[1]{{\left\{#1\right\}}}
\newcommand{\Angle}[1]{{\left\langle#1\right\rangle}}
\newcommand{\pipe}[1]{{\left|#1\right|}}

\newcommand{\harpoon}{\overset{\rightharpoonup}}

\renewcommand{\thesection}{\Roman{section}} 

\title{Hartshorne Exercises}
\author{Jas Singh}
\date{}

\begin{document}

\maketitle

\tableofcontents
\newpage

\section{Varieties}
    \begin{conventions}
        $k$ is an algebraically closed field.
    \end{conventions}
    
     \subsection{Affine Varieties}
         \begin{conventions}
             $A = k[x_1, \dots, x_n]$.
         \end{conventions}
 
 	     \subsubsection{INCOMPLETE} \subfile{./chapter 1/section 1/problem 01}
         \subsubsection{} \subfile{./chapter 1/section 1/problem 02}
         \subsubsection{} \subfile{./chapter 1/section 1/problem 03}
         \subsubsection{} \subfile{./chapter 1/section 1/problem 04}
         \subsubsection{} \subfile{./chapter 1/section 1/problem 05}
         \subsubsection{} \subfile{./chapter 1/section 1/problem 06}
         \subsubsection{} \subfile{./chapter 1/section 1/problem 07}
         \subsubsection{} \subfile{./chapter 1/section 1/problem 08}
         \subsubsection{} \subfile{./chapter 1/section 1/problem 09}
         \subsubsection{} \subfile{./chapter 1/section 1/problem 10}
         \subsubsection{} \subfile{./chapter 1/section 1/problem 11}
         \subsubsection{} \subfile{./chapter 1/section 1/problem 12}
      \newpage
    
    \subsection{Projective Varieties}
        \begin{conventions}
            $S = k[x_0, \dots, x_n]$.
        \end{conventions}
        
        \subsubsection{} \subfile{./chapter 1/section 2/problem 01}
        \subsubsection{} \subfile{./chapter 1/section 2/problem 02}
        \subsubsection{} \subfile{./chapter 1/section 2/problem 03}
        \subsubsection{} \subfile{./chapter 1/section 2/problem 04}
        \subsubsection{} \subfile{./chapter 1/section 2/problem 05}
        \subsubsection{} \subfile{./chapter 1/section 2/problem 06}
        \subsubsection{} \subfile{./chapter 1/section 2/problem 07}
        \subsubsection{} \subfile{./chapter 1/section 2/problem 08}
        \subsubsection{INCOMPLETE} \subfile{./chapter 1/section 2/problem 09}
        \subsubsection{} \subfile{./chapter 1/section 2/problem 10}
        \subsubsection{} \subfile{./chapter 1/section 2/problem 11}
        \subsubsection{} \subfile{./chapter 1/section 2/problem 12}
        \subsubsection{} \subfile{./chapter 1/section 2/problem 13}
        \subsubsection{} \subfile{./chapter 1/section 2/problem 14}
        \subsubsection{} \subfile{./chapter 1/section 2/problem 15}
        \subsubsection{} \subfile{./chapter 1/section 2/problem 16}
        \subsubsection{} \subfile{./chapter 1/section 2/problem 17}
\newpage

\printbibliography

\end{document}
